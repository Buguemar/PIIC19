\section{Introduction}

There is not a clear unique way to detect and confirm exoplanets.
%This is because planets emit or reflect are very dim light magnitudes compared to their hosting stars and they are very near to the host star compared to the observation distance. 
%Therefore, it is a extremely difficult task, with only a couple of dozen exoplanets directly photographed while the majority of known exoplanets have been detected using indirect methods.
%The most successful indirect techniques study some variations on the host star product of its orbiting planet. For example,
Besides a couple of dozen exoplanets that have been directly photographed, analyzing the \textit{transit photometry}
%, used to analyze the light intensity (photometric observation)
and \textit{radial velocity} of the hosting star are the two more successful indirect observational methods for detecting exoplanets. However, the variations that an orbiting planet produces in its host could be very mild compared to other complex processes that a system may exhibit. 

Fortunately, technological advances in instrumentation and photometry have allowed experiments like the Kepler mission to have enough sensitivity for detecting a large number of exoplanets.
NASA has reported that more than 4000 exoplanet has been detected\footnote{http://exoplanets.nasa.gov} by grounded or spatial observatories, with $75\%$ of them using transit photometry. However, detecting an exoplanet from light curves is a comprehensive task that was primarily achieved through time-consuming manual processes. 
The increasing popularity of Machine Learning techniques and the large amount of data generated by observatories, allow the automatic learning of models that help astronomers to detect exoplanets by reducing effort and time. 
The common approach used to classify a light curve is to extract hand-crafted \textit{specialized features} from it and apply classic machine learning methods. For example, \citet{richards2011machine} present specialized features for variable star detection for the Catalina Real-Time Transient Survey (CRTS) and the Kepler Mission by \citet{donalek2013feature} using $k$-NN and Decision Tree models.
Similarly, \citet{hinners2017machine} use these features in order to predict the Kepler features using classic machine learning methods. The alternative is to use problem-learned representations, such as \citet{bugueno2018refining} introduce for the detection of exoplanet on the same Kepler dataset.

Deep Learning has been recently applied to face astronomy problems. In this domain, convolutional neural networks (CNNs) have been the most popular technique. For instance, \citet{shallue2018identifying} use a 1D convolutional neural network (CNN) model (\textit{Astronet}) to classify exoplanets on Kepler Mission with a global and local representation of the folded light curve. In the same vein, \citet{schanche2019machine} use a 1D CNN network to detect exoplanets among variable stars (4 classes) on WASP mission, and \citet{osborn2020rapid} use a 1D CNN network combined with scientific domain knowledge \citep{ansdell2018scientific} (\textit{Exonet}) on the recent TESS mission.
However, even though Machine Learning models are very helpful, most of the methods rely on hand-crafted features and external metadata to cope with the task. Therefore, a thorough and detailed analysis of the light curves only occurs once the metadata has been collected. 

This article focuses on detecting transients on light curves based only on the raw light curve data, in order to isolate the source of information. Cross-matched metadata is certainly important, but including it in the evaluation only hinders the effectiveness of the models used to tackle the time series complexity. The key idea of this article is to harness the power of Deep Learning for image processing, specifically 2D convolutional neural networks, and apply it to unevenly-sampled time series.
Concretely, we transform the light curves into a 2-channel image representation using and adapted version of Markov Transition Field (MTF) technique \citep{wang2015imaging}.
The first channel corresponds to the unevenly-sampled light curve measurements and the second channel represents the time information from the curve. The unevenly-sampled problem is addressed by defining semi-continuous transitions where each one represents measurements with a delta time below a maximum value.
This allows to extract temporal information from the raw light curve measurements. 

We assess our models using data from the Kepler mission, i.e. Kepler Objects of Interest (KOI\footnote{\texttt{http://archive.stsci.edu/search fields.php?mission=kepler\_koi}}) dataset. 
The results of our experiments show that we reach competitive performance and speed in the task of detecting exoplanets compared to using hand-crafted specialized features and similar deep learning approaches. 
%Our work achieves a macro-averaged F1-score of 77\%; 84\% in False Positive and 70\% in Confirmed classes. 
%which is obtained from a new curve which we call Sample Detection Curve (SDC) which counts the consecutive valid sample measures 

The article is organized as follows. Section 2 provides the background of the exoplanet detection problem with a discussion of the related work. After this, our proposal is introduced in Section 3. The experimental setting is presented in Section 4 with the corresponding results in Section 5.
An exploratory analysis of the method is discussed in Section 6 to finally conclude in Section 7.
%Finally, in Section 7 we conclude commenting on the extensions of our proposal.




%\section{Background}
%In astronomy exists different types of data that are processed, studied and analyzed, i.e. catalogs, time series and data cubes. 
%In the exoplanet field the data used are time series, particularly light curves. 
%A \textit{light curve} is defined as a graph of light intensity from a celestial object or region as a function of time, which can be obtained usually from photometric observation of a particular optical band. 
%Working with time series on astronomy has the challenge that they are quite uneven on their sampling, meaning that there are missing values from several timestamps or that measurements are not acquired uniformly.
%Some factors that produce these problems mostly affect ground observatories such as cloudy weather or Earth rotation.

%subsection*{Light Curves}
%When the light intensity of a star (luminosity magnitude) is measured, the resulting time series could varies its values as a result of the star composition or because of an orbiting planet. 
%The first case correspond to variable stars, which can be produced depending on the chemical composition of the star or because a celestial body (non necessarily planets) blocks the light. 
%By the other hand, if a planet is involved, it passes in front of the star and a fraction of the light is blocked (as an eclipse), this phenomenon is called \textbf{transit} and is an effective method to find exoplanets.

%existe otro?? SUPERNOVAS, MICROLENSING